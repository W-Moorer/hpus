% !TeX program = xelatex
\documentclass[UTF8,a4paper,12pt]{ctexart}

\usepackage{geometry}
\geometry{margin=2.2cm}
\usepackage{amsmath,amssymb,amsthm,bm,mathtools}
\usepackage{enumitem}
\usepackage{booktabs}
\usepackage{algorithm}
\usepackage{algpseudocode}
\usepackage{hyperref}
\hypersetup{colorlinks=true,linkcolor=blue,urlcolor=blue,citecolor=blue}

% Theorem environments
\newtheorem{theorem}{定理}
\newtheorem{lemma}{引理}
\newtheorem{proposition}{命题}
\newtheorem{corollary}{推论}
\theoremstyle{definition}
\newtheorem{definition}{定义}
\theoremstyle{remark}
\newtheorem{remark}{注记}

% Convenience macros
\newcommand{\R}{\mathbb{R}}
\newcommand{\eps}{\varepsilon}
\newcommand{\dd}{\mathrm{d}}
\newcommand{\norm}[1]{\left\lVert #1 \right\rVert}
\newcommand{\ip}[2]{\left\langle #1,#2 \right\rangle}
\newcommand{\supp}{\operatorname{supp}}
\newcommand{\dist}{\operatorname{dist}}
\newcommand{\argmin}{\operatorname*{arg\,min}}
\newcommand{\softmin}{\operatorname{smin}}

\title{面向接触仿真的分区 HRBF--PoU 隐式曲面重建\\
\large ——基于拓扑邻接的局部 Soft-Min 尖锐特征平滑连接(初稿)}
\author{(作者信息待补充)}
\date{\today}

\begin{document}
\maketitle

\begin{abstract}
本文提出一种面向接触仿真的全局隐式曲面构建方法:首先利用二面角阈值将三角网格切割为若干拓扑光滑区域(regions),在每个区域内部采用局部 Hermite RBF(HRBF)拟合并通过严格 $C^2$ 的分割统一(Partition of Unity, PoU)权重进行高阶连续拼接,得到每个区域的隐式场 $F_i(x)$。针对尖锐边/角点等非光滑特征,本文不采用全局混合,而是提出“拓扑邻接驱动”的局部 Soft-Min(log-sum-exp)连接机制:仅当查询点靠近\emph{主区域}的尖锐边界时,才将其拓扑相邻区域纳入候选集,并在一条可控宽度 $h$ 的管状邻域内实现平滑过渡。该设计在齿轮齿间狭缝等复杂构型中有效避免“空间近但拓扑不相邻”导致的错误融合(虚假桥接),同时可高效提供接触所需的嵌入深度与法向量(由 $\nabla F$ 给出)。文中给出完整的数学定义、光滑性与局部性条件、变量软化尺度 $\eps(x)$ 下的梯度闭式表达,以及数值稳定实现与加速策略。
\end{abstract}

\section{引言}
在刚体/柔体接触仿真中,隐式表示(implicit surface)因其可直接提供接触深度与法向信息而具有优势。理想情况下,我们希望获得接近有符号距离场(Signed Distance Field, SDF)的隐式函数 $F(x)$,使得:
\begin{itemize}[leftmargin=2em]
\item 接触深度可近似为 $d(x)\approx F(x)/\norm{\nabla F(x)}$,或通过少量 Newton 迭代精化;
\item 法向可由 $\bm n(x)=\nabla F(x)/\norm{\nabla F(x)}$ 稳定得到;
\item 在几何光滑区域内具有高阶连续性(至少 $C^2$),以减少接触力抖动;
\item 能处理包含尖锐边、角点的复杂 CAD/三角网格几何。
\end{itemize}

然而,尖锐特征与复杂构型(例如齿轮齿间狭缝)会带来两类典型困难:\textbf{(i)} 几何在尖锐边处法向不连续,若强行全局高阶光滑拟合会产生过度圆角;\textbf{(ii)} 在空间上相互接近但拓扑上不相邻的曲面片(例如狭缝两侧)会导致“包围域/支撑域重叠”,若以空间重叠为融合依据将出现错误桥接,从而破坏接触检测。
本文方法的核心思想是:\emph{区域内高阶、区域间仅在拓扑允许的尖锐边界邻域内平滑连接}。这使得模型在大部分区域保持精确与稳定,在尖锐处引入可控的小尺度圆角,从而适用于高精度接触仿真。

\section{问题设定与记号}
给定定向三角网格 $\mathcal M=(V,\mathcal T)$,其中 $V=\{v_m\in\R^3\}$,$\mathcal T$ 为三角面集合。
设网格几何曲面为 $\Sigma\subset\R^3$。

\subsection{目标}
构造全局隐式函数 $F:\R^3\to\R$,其零水平集
\[
\mathcal S=\{x\in\R^3\mid F(x)=0\}
\]
逼近 $\Sigma$。并支持接触查询:
\[
\bm n(x)=\frac{\nabla F(x)}{\norm{\nabla F(x)}},\qquad
d(x)\approx \frac{F(x)}{\norm{\nabla F(x)}}.
\]

\subsection{几何查询与 SDF 约束(实现背景)}
对于任意点 $x$,定义最近点映射
\[
c(x)=\argmin_{y\in\Sigma}\norm{x-y},\qquad d_0(x)=\norm{x-c(x)}.
\]
通过 AABB/BVH 可高效返回最近三角形 $t^*(x)\in\mathcal T$ 与最近点 $c(x)$。符号由伪法向(pseudo-normal)或绕数(winding number)给出 $s(x)\in\{-1,+1\}$,从而有符号距离(理想 SDF)
\[
\phi(x)=s(x)\,d_0(x).
\]
本文使用 $\phi(x)$ 作为拟合数据源(值约束)并可在曲面邻域用网格法向近似 $\nabla\phi$(导数约束)。

\section{基于尖锐边的区域分解}
\subsection{尖锐边检测}
对共享边 $e$ 的相邻面 $t_a,t_b$,设其单位法向为 $n_a,n_b$,二面角
\[
\theta(e)=\arccos(n_a^\top n_b).
\]
给定阈值 $\theta_0$,定义尖锐边集合
\[
\Gamma=\{e\mid \theta(e)>\theta_0\}.
\]

\subsection{区域(regions)与尖锐边界}
沿 $\Gamma$ 切割网格,在“非尖锐边连通”意义下得到面连通分量
\[
\mathcal T=\bigsqcup_{i=1}^{N} R_i,
\]
其中 $R_i$ 是第 $i$ 个光滑区域。

\begin{definition}[区域尖锐边界与对侧映射]
对每个区域 $R_i$,定义其\emph{尖锐边界集合}
\[
\Gamma_i:=\{e\in\Gamma\mid e\subset \partial R_i,\ \text{且跨过 }e\text{ 进入另一 }R_j\}.
\]
对每条 $e\in\Gamma_i$,定义对侧区域映射 $\mathrm{opp}(i,e)=j$。
\end{definition}

\begin{remark}
后文用于尖锐处连接的候选集合必须由 $\Gamma_i$(而非空间包围域)触发。该设计是避免齿轮齿间狭缝等“空间近但拓扑不相邻”造成误融合的关键。
\end{remark}

\section{区域内 HRBF--PoU 隐式场构建}
本节在每个区域内部构造一个高阶连续隐式场 $F_i(x)$。为了数学完备性与工程可控性,我们采用:(i)patch 覆盖 + 有限重叠;(ii)严格 $C^2$ 紧支撑权重形成 PoU;(iii)局部 HRBF(值 + 法向方向导数约束)拟合;(iv)patch 常数移位对齐减少漂移。

\subsection{patch 覆盖与有限重叠公理}
对每个区域 $i$,选取其重建域 $\Omega_i\subset\R^3$(开集,通常为区域曲面附近的管状邻域)。假设存在 patch 开覆盖 $\{\Omega^{\mathrm c}_{im}\}_{m=1}^{M_i}$ 满足:

\textbf{公理 A2(覆盖与有限重叠)}
\[
\Omega_i\subset\bigcup_{m=1}^{M_i}\Omega^{\mathrm c}_{im},\qquad
\#\{m:x\in\Omega^{\mathrm c}_{im}\}\le K,\ \forall x\in\Omega_i.
\]
并采用\emph{双半径}:权重半径 $\rho^{\mathrm w}_{im}$ 与覆盖半径 $\rho^{\mathrm c}_{im}>\rho^{\mathrm w}_{im}$。

\subsection{椭球/OBB patch(推荐)}
为抑制硬边两侧跨边混合,优先使用椭球 patch:
\[
q_{im}(x)=(x-\xi_{im})^\top Q_{im}(x-\xi_{im}),\qquad Q_{im}\succ0.
\]
定义覆盖域与权重支撑域:
\[
\Omega^{\mathrm c}_{im}=\{x:q_{im}(x)<(\rho^{\mathrm c}_{im})^2\},\qquad
\Omega^{\mathrm w}_{im}=\{x:q_{im}(x)\le(\rho^{\mathrm w}_{im})^2\}\subset\Omega^{\mathrm c}_{im}.
\]
在薄壁/狭缝附近,可令 $Q_{im}$ 在法向方向更“薄”,以减少跨缝覆盖(仅用于加速/稳健性;理论正确性由后文拓扑候选集保证)。

\subsection{严格 $C^2$ 紧支撑 bump 与 PoU 权重}
定义 $C^2$ 紧支撑 bump:
\[
\rho(t)=
\begin{cases}
(1-t)^4(4t+1), & 0\le t<1,\\
0, & t\ge 1.
\end{cases}
\]
满足 $\rho\in C^2([0,\infty))$ 且 $\rho(1)=\rho'(1)=\rho''(1)=0$。

定义未归一化权重:
\[
\tilde w_{im}(x)=\rho\!\left(\frac{q_{im}(x)}{(\rho^{\mathrm w}_{im})^2}\right)\ge 0,\qquad
S_i(x)=\sum_{k=1}^{M_i}\tilde w_{ik}(x),
\]
并定义 PoU 权重:
\[
w_{im}(x)=\frac{\tilde w_{im}(x)}{S_i(x)}.
\]

\textbf{公理 A3-A(分母正下界)}\quad 存在常数 $c_w>0$ 使得
\[
S_i(x)\ge c_w,\qquad \forall x\in\Omega_i.
\]
工程上可通过加密 patch 中心、增大覆盖半径 $\rho^{\mathrm c}$ 或修补覆盖空洞保证该条件成立。

\begin{lemma}[PoU 权重的 $C^2$ 性]
若 $q_{im}$ 为多项式(或至少 $C^2$),$\rho\in C^2$ 且 (A3-A) 成立,则 $w_{im}\in C^2(\Omega_i)$ 且 $\sum_m w_{im}\equiv 1$。
\end{lemma}

\subsection{局部 HRBF:值 + 法向方向导数约束(标量形式)}
硬边处法向不连续。因此每个 patch 仅从单一光滑区域 $R_i$ 内采样与拟合,避免跨硬边的矛盾导数约束。

对 patch 内 $N$ 个约束点 $(x_j,n_j)$,令 $r_j(x)=\norm{x-x_j}$,选定标量径向核 $\phi(r)$,并引入一次多项式基
\[
p(x)=[1,x,y,z]^\top.
\]

采用“源点法向方向导数基”的 HRBF 表示:
\[
\boxed{
s_{im}(x)=\sum_{j=1}^{N}\alpha_j\,\phi(r_j(x))
-\sum_{j=1}^{N}\eta_j\big(n_j\cdot\nabla \phi(r_j(x))\big)
+c^\top p(x).
}
\]
未知量为 $\alpha\in\R^N$、$\eta\in\R^N$、$c\in\R^4$。该形式等价于经典 HRBF 中将向量系数限制为 $\beta_j=\eta_j n_j$,从 $3N$ 降维到 $N$,更鲁棒且更易稳定。

\paragraph{核导数闭式($r>0$)}
令 $u=(x-y)/r$,则
\[
\nabla_x\phi(r)=\phi'(r)\,u,
\qquad
H_x\phi(r)=\phi''(r)uu^\top+\frac{\phi'(r)}{r}(I-uu^\top).
\]

\paragraph{约束条件}
给定值目标 $f_i$(通常靠近曲面取 $0$,或取采样 SDF 值),法向方向导数目标 $g_i$(可取 patch 尺度常数或来自 SDF 梯度),施加:
\[
s_{im}(x_i)=f_i,\qquad n_i\cdot\nabla s_{im}(x_i)=g_i.
\]

\paragraph{线性系统(对称块结构)}
定义矩阵块($i$ 为约束行,$j$ 为中心列):
\[
\Phi_{ij}=\phi(\norm{x_i-x_j}),\quad
G_{ij}=n_j\cdot\nabla\phi(\norm{x_i-x_j}),\quad
D_{ij}=n_i\cdot\nabla\phi(\norm{x_i-x_j}),
\]
\[
H_{ij}=n_i^\top H_x\phi(\norm{x_i-x_j})n_j.
\]
多项式块:
\[
P_{i,:}=p(x_i)^\top=[1,x_i,y_i,z_i],\qquad
R_{i,:}=(n_i\cdot\nabla p(x_i))^\top=[0,n_{i,x},n_{i,y},n_{i,z}].
\]
则未知量 $y=[\alpha;\eta;c]$ 满足
\[
\boxed{
\begin{bmatrix}
\Phi & -G & P\\
D & -H & R\\
P^\top & R^\top & 0
\end{bmatrix}
\begin{bmatrix}
\alpha\\ \eta\\ c
\end{bmatrix}
=
\begin{bmatrix}
f\\ g\\ 0
\end{bmatrix}.
}
\]
可在主块加 ridge 正则以抑制噪声与病态:
\[
\Phi\leftarrow \Phi+\lambda_\alpha I,\qquad H\leftarrow H+\lambda_\eta I.
\]

\subsection{patch 常数移位对齐}
局部场在导数约束占主导时可能产生 patch 间常数漂移。采用常数移位:
\[
\mu_{im}=\frac{1}{n_{im}}\sum_{j=1}^{n_{im}} s_{im}(x_j^{(m)}),\qquad
\tilde s_{im}(x)=s_{im}(x)-\mu_{im},
\]
以改善 overlap 区一致性。常数平移不改变光滑阶数。

\subsection{区域隐式场 $F_i$ 的 PoU 拼接}
定义
\[
\boxed{
F_i(x)=\sum_{m=1}^{M_i} w_{im}(x)\,\tilde s_{im}(x),\qquad x\in\Omega_i.
}
\]
由 $w_{im},\tilde s_{im}\in C^2$ 可得 $F_i\in C^2(\Omega_i)$。

\section{区域有效域门控与惩罚延拓}
为避免某区域在其可信域外参与竞争,构造 $C^2$ 门控 $\chi_i(x)\in[0,1]$,使得
\[
\chi_i(x)=1\ \text{于可信核心域}\ \Omega_i^{\text{core}},\qquad
\chi_i(x)=0\ \text{于}\ \R^3\setminus \Omega_i.
\]
定义惩罚延拓:
\[
\boxed{
\tilde F_i(x)=F_i(x)+\lambda(1-\chi_i(x)),
}
\]
其中 $\lambda\gg 0$。其作用是:即便域外被纳入候选集,指数权重也会被强烈压制,从而不会产生“融合过深”。

\section{全局隐式场:拓扑邻接驱动的局部 Soft-Min}
本节给出全局隐式 $F(x)$ 的数学定义,并证明其可避免齿轮狭缝等复杂构型下的误融合。

\subsection{主区域选择:由最近三角形确定}
通过 BVH 得到最近三角形
\[
t^*(x)=\argmin_{t\in\mathcal T}\dist(x,t),
\qquad
i_0(x)=\mathrm{region}(t^*(x)).
\]
\begin{remark}
当 $x$ 位于中轴(medial axis)上时最近三角形可能不唯一,该集合在 $\R^3$ 中为零测度。真实 SDF 在中轴处本就不可微,本文主要关注曲面邻域与接触查询区域,通常可忽略该集合,或采用固定 tie-break 规则定义 $i_0(x)$。
\end{remark}

\subsection{只允许“主区域尖锐边界”触发跨区连接}
对主区域 $i_0$,考虑其尖锐边界集合 $\Gamma_{i_0}$。进一步将其按对侧区域分组:
\[
\Gamma_{i_0\to j}=\{e\in\Gamma_{i_0}\mid \mathrm{opp}(i_0,e)=j\}.
\]
定义到该边界组的距离
\[
\delta_{i_0\to j}(x)=\dist(x,\Gamma_{i_0\to j})=\min_{e\in\Gamma_{i_0\to j}}\dist(x,e).
\]
给定圆角带宽 $h>0$,定义边界门控
\[
\beta_{i_0\to j}(x)=\rho\!\left(\frac{\delta_{i_0\to j}(x)}{h}\right)\in[0,1].
\]
注意:当 $\delta\ge h$ 时 $\beta=0$ 且在 $\delta=h$ 处满足二阶导为 0 的平坦边界性质。

\subsection{候选集合与“指数惩罚门控”}
令主区域的拓扑相邻集合为
\[
\mathrm{Adj}(i_0)=\{j\mid \Gamma_{i_0\to j}\neq\emptyset\}.
\]
本文将候选集合固定为
\[
C(x)=\{i_0(x)\}\cup \mathrm{Adj}(i_0(x)),
\]
并通过能量惩罚确保\emph{仅在尖锐边邻域}相邻区域才起作用。

定义候选能量:
\[
E_{i_0}(x)=\tilde F_{i_0}(x),
\]
\[
E_j(x)=\tilde F_j(x)+\Lambda\big(1-\beta_{i_0\to j}(x)\big),\qquad j\in\mathrm{Adj}(i_0),
\]
其中 $\Lambda\gg 0$。当 $x$ 远离 $i_0$ 的边界组 $\Gamma_{i_0\to j}$ 时,$\beta=0$,于是 $E_j=\tilde F_j+\Lambda$,其在 soft-min 中的影响被指数级压制。

\subsection{变量软化尺度 \texorpdfstring{$\eps(x)$}{eps(x)}}
定义
\[
\beta_{\max}(x)=\max_{j\in \mathrm{Adj}(i_0(x))}\beta_{i_0(x)\to j}(x),
\]
\[
\boxed{
\eps(x)=\eps_{\mathrm{far}}+(\eps_{\mathrm{edge}}-\eps_{\mathrm{far}})\,\beta_{\max}(x),
\qquad 0<\eps_{\mathrm{far}}\ll \eps_{\mathrm{edge}}\sim O(h).
}
\]

\subsection{全局隐式函数定义(Soft-Min / log-sum-exp)}
\[
\boxed{
F(x)= -\eps(x)\,\log\!\sum_{k\in C(x)}\exp\!\left(-\frac{E_k(x)}{\eps(x)}\right).
}
\]

\begin{proposition}[区域内主控与指数抑制]
若对所有 $j\in\mathrm{Adj}(i_0(x))$ 有 $\delta_{i_0\to j}(x)\ge h$,则 $\beta_{i_0\to j}(x)=0$,从而
\[
F(x)=\tilde F_{i_0}(x)-\eps(x)\log\!\left(1+\sum_{j\in\mathrm{Adj}(i_0)}\exp\!\left(-\frac{\tilde F_j(x)+\Lambda-\tilde F_{i_0}(x)}{\eps(x)}\right)\right).
\]
特别地,当 $\Lambda/\eps(x)\to\infty$ 时(或在工程上取足够大),有
\[
F(x)=\tilde F_{i_0}(x)+O\!\left(e^{-\Lambda/\eps(x)}\right),
\]
即主区域在边界邻域外严格主导,且相邻区域影响以指数速度消失。
\end{proposition}

\begin{proposition}[避免复杂构型误融合:以齿轮齿间狭缝为例]
设 $x$ 位于齿间狭缝,使得 $x$ 同时落入多个区域的空间包围域/patch 支撑域内,但狭缝两侧曲面\emph{拓扑不相邻}(不共享尖锐边界)。则本文定义的候选集合 $C(x)$ 不会包含狭缝对侧的非邻接区域,因而不会产生“跨缝圆角桥接”。当 $x$ 在狭缝中轴附近发生最近三角形切换时,$i_0(x)$ 会切换,这对应真实距离场的最近面切换(几何事实),而不是错误融合。
\end{proposition}

\begin{remark}
上述命题表明:包围域重叠只用于加速,不作为融合依据。真正决定融合的仅是“主区域尖锐边界邻域”(通过 $\Gamma_{i_0}$ 与 $\beta_{i_0\to j}$)。
\end{remark}

\section{梯度、法向与接触深度:完整闭式表达}
\subsection{数值稳定的 log-sum-exp}
当 $h$ 与 $\eps$ 极小(如 $10^{-7}\!\sim\!10^{-5}$)时,直接计算 $\exp(-E/\eps)$ 易下溢。设
\[
a_k(x)=-\frac{E_k(x)}{\eps(x)},\qquad a_{\max}(x)=\max_{k\in C(x)} a_k(x),
\]
则
\[
\log\sum_{k\in C}\exp(a_k)
=
a_{\max}+\log\sum_{k\in C}\exp(a_k-a_{\max}),
\]
可稳定计算 $F(x)$ 与 soft 权重。

\subsection{soft 权重}
定义
\[
\pi_k(x)=\frac{\exp\!\left(-E_k(x)/\eps(x)\right)}{\sum_{\ell\in C(x)}\exp\!\left(-E_\ell(x)/\eps(x)\right)},
\qquad \sum_{k\in C}\pi_k=1.
\]

\subsection{变量 \texorpdfstring{$\eps(x)$}{eps(x)} 下的完整梯度}
记
\[
Z(x)=\sum_{k\in C(x)}\exp\!\left(-\frac{E_k(x)}{\eps(x)}\right),\qquad
F(x)=-\eps(x)\log Z(x).
\]
在 $i_0(x)$ 固定(远离中轴)且 $C(x)$ 固定的区域内,可得:

\begin{theorem}[完整梯度公式(含 $\nabla\eps$ 项)]
设所有 $E_k,\eps$ 在某开集内为 $C^1$ 且 $\eps>0$,则
\[
\boxed{
\nabla F(x)=\sum_{k\in C(x)}\pi_k(x)\,\nabla E_k(x)
+\frac{\nabla\eps(x)}{\eps(x)}\left(\sum_{k\in C(x)}\pi_k(x)E_k(x)+F(x)\right).
}
\]
\end{theorem}

\paragraph{$\nabla E_k$ 的组成}
主区域 $k=i_0$:
\[
\nabla E_{i_0}=\nabla \tilde F_{i_0}=\nabla F_{i_0}-\lambda\nabla\chi_{i_0}.
\]
相邻区域 $k=j$:
\[
\nabla E_j=\nabla \tilde F_j-\Lambda\nabla\beta_{i_0\to j}.
\]
其中
\[
\nabla F_i(x)=\sum_m\Big((\nabla w_{im})\,\tilde s_{im}+w_{im}\,\nabla\tilde s_{im}\Big),
\]
且
\[
\nabla w_{im}=\frac{\nabla \tilde w_{im}}{S_i}-\frac{\tilde w_{im}}{S_i^2}\nabla S_i,\qquad
\nabla S_i=\sum_k \nabla\tilde w_{ik}.
\]

\subsection{法向与深度}
法向:
\[
\bm n(x)=\frac{\nabla F(x)}{\norm{\nabla F(x)}}.
\]
一阶深度近似:
\[
d(x)\approx \frac{F(x)}{\norm{\nabla F(x)}}.
\]
为获得更高精度,可做 1--2 次 Newton 投影到零水平集:
\[
x_{t+1}=x_t-\frac{F(x_t)}{\norm{\nabla F(x_t)}^2}\nabla F(x_t).
\]

\section{算法流程与复杂度}
\subsection{预处理与查询伪代码}
\begin{algorithm}[h]
\caption{预处理:分区 HRBF--PoU 与拓扑邻接结构}
\begin{algorithmic}[1]
\State 输入网格 $\mathcal M=(V,\mathcal T)$,阈值 $\theta_0$,参数 $h,\eps_{\mathrm{far}},\eps_{\mathrm{edge}},\lambda,\Lambda$
\State 计算二面角,得尖锐边 $\Gamma$
\State 沿 $\Gamma$ 切割面集,得 regions $\{R_i\}$
\State 构造每个 region 的尖锐边界集合 $\Gamma_i$ 与对侧映射 $\mathrm{opp}(i,e)$
\State 构造 mesh BVH(最近三角形/最近点),以及每个 $\Gamma_i$ 的边界距离查询结构(edge BVH)
\For{每个 region $i$}
  \State 设计重建域 $\Omega_i$ 与 patch 覆盖 $\{\Omega^{\mathrm c}_{im}\}$(满足 A2)
  \State 在每个 patch 内采样约束点与法向(不跨硬边)
  \State 解局部 HRBF 得 $s_{im}$,做常数移位得 $\tilde s_{im}$
  \State 构造 $C^2$ 权重 $w_{im}$(满足 A3-A),得 $F_i=\sum_m w_{im}\tilde s_{im}$
  \State 构造门控 $\chi_i$ 并实现惩罚延拓 $\tilde F_i$
\EndFor
\State 输出:$\{F_i,\tilde F_i\}$、BVH、$\{\Gamma_i,\mathrm{opp}\}$
\end{algorithmic}
\end{algorithm}

\begin{algorithm}[h]
\caption{查询:全局隐式 $F(x)$ 与法向/深度}
\begin{algorithmic}[1]
\State 输入点 $x$
\State 通过 mesh BVH 得最近三角形 $t^*(x)$ 与主区域 $i_0$
\State 对 $j\in \mathrm{Adj}(i_0)$ 计算 $\delta_{i_0\to j}(x)$,并得 $\beta_{i_0\to j}(x)$
\State 计算 $\eps(x)=\eps_{\mathrm{far}}+(\eps_{\mathrm{edge}}-\eps_{\mathrm{far}})\max_j\beta_{i_0\to j}(x)$
\State 计算候选能量 $E_{i_0}$ 与 $E_j=\tilde F_j+\Lambda(1-\beta_{i_0\to j})$
\State 用稳定 log-sum-exp 计算 $F(x)$ 与 $\pi_k(x)$
\State 由完整梯度公式计算 $\nabla F(x)$,输出 $\bm n=\nabla F/\norm{\nabla F}$,深度 $d\approx F/\norm{\nabla F}$
\end{algorithmic}
\end{algorithm}

\subsection{复杂度讨论}
\begin{itemize}[leftmargin=2em]
\item mesh BVH 最近三角形查询:$O(\log|\mathcal T|)$;
\item 边界距离查询:对固定主区域 $i_0$,只需查询 $\Gamma_{i_0}$(或其分组 BVH),复杂度 $O(\log|\Gamma_{i_0}|)$;
\item region 内 $F_i$ 评估:通过 patch 索引取 Top-$K$ patch,仅需 $O(K)$ 次 HRBF/权重评估;
\item soft-min 仅在主区域邻接集合上进行,通常候选数很小(CAD 模型常为 2,角点为 3--6)。
\end{itemize}

\section{讨论与局限}
\begin{itemize}[leftmargin=2em]
\item \textbf{中轴处不可微性:} 最近三角形切换对应真实距离场的不可微集合(中轴),这并非算法缺陷。接触仿真通常在曲面附近工作且可 warm-start,影响可控。
\item \textbf{参数选择:} $h$ 决定圆角带宽(你可设 $10^{-7}\sim10^{-5}$),$\eps_{\mathrm{edge}}\sim O(h)$,$\eps_{\mathrm{far}}\ll \eps_{\mathrm{edge}}$;$\Lambda$ 需满足 $\Lambda/\eps_{\mathrm{far}}\gg 1$ 以保证指数抑制;$\lambda$ 由域外可信度决定。
\item \textbf{采样策略:} 靠近硬边处可降低导数约束强度或仅施加值约束,以避免硬边附近的法向估计噪声对拟合造成放大。
\end{itemize}

\section{结论与后续工作}
本文给出一种面向接触仿真的分区隐式曲面重建框架:区域内 HRBF--PoU 提供高阶连续与高精度拟合;区域间采用拓扑邻接驱动的局部 soft-min,仅在尖锐边界邻域进行平滑连接,并通过指数惩罚门控保证在复杂构型(如齿轮狭缝)中不发生错误融合。后续工作包括:更系统的参数自适应(按局部曲率与接触尺度设置 $h,\eps$)、在动态仿真中的 warm-start/缓存策略,以及在非闭合网格与多材料接触下的拓展。

\appendix
\section{附录:完整梯度推导(简要)}
令 $F=-\eps\log Z$,$Z=\sum_k \exp(-E_k/\eps)$。则
\[
\nabla F=-(\nabla\eps)\log Z-\eps\nabla(\log Z),\qquad
\nabla(\log Z)=\sum_k \pi_k \nabla\left(-\frac{E_k}{\eps}\right).
\]
且
\[
\nabla\left(-\frac{E_k}{\eps}\right)=-\frac{\nabla E_k}{\eps}+\frac{E_k}{\eps^2}\nabla\eps.
\]
代入并整理即得正文定理中的闭式表达:
\[
\nabla F=\sum_k\pi_k\nabla E_k+\frac{\nabla\eps}{\eps}\left(\sum_k\pi_k E_k+F\right).
\]

\end{document}
